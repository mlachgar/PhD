%
% contenu du fichier : intro.tex
%
\chapter*{Abstract}
% pour faire apparaitre l'introduction dans le sommaire et que les minitocs soient au bon
% endroit
%\addstarredchapter{Introduction générale}  
% Pour que l'entete soit correcte car chapter* ne redefinit pas l'entete.
%\markboth{INTRODUCTION}{}
\addstarredchapter{Abstract} 
Complex networks is a fundamental field of research that model and study artificial and natural networks in our real world. The discovery of common universal properties, almost to all real networks, such as small-world property and scale-free distribution, has revolutionized the way as these networks are studied, modeled and processed.
 A complex network consists of a large number of interacting entities, hence the emergence of properties at large-scale. The object of this thesis is to introduce some contributions to this domaine. In the beginning, we present the state of the art necessary to the readers, next we analyze the model of Barabási and Albert relying on a new model having the complement of the probability of the first one. After, we study uncorrelated scale-free networks, based on simple steps and assumptions. Then, we treat Newman and Watts' small-world model by relying on renormalization group transformation. 
 Finally, we propose a method to predict the type of phase transitions in the percolation phenomenon in the vicinity of the critical point.