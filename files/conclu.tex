
\chapter*{Conclusion}
% pour faire apparaitre l'introduction dans le sommaire
\addcontentsline{toc}{chapter}{Conclusion}
 
% Pour que l'entete soit correcte car chapter* ne redefinit pas l'entete.
\markboth{CONCLUSION}{}


Le travail présenté dans cette thèse constitue une contribution au domaine des réseaux complexes ainsi que plusieurs aspects fondamentaux ont été abordés dans le sujet.\\
Nous avons effectué des développements théoriques originaux principalement dans le cadre de la théorie du champ moyen. Ces traitement théoriques ont été toujours comparés aux simulations numériques dont les algorithmes de base ont été suffisamment optimisés. Nous pouvons résumer nos contributions comme suit: \\
Dans le deuxième chapitre, nous avons introduit un modèle de croissance de réseau complexe avec une probabilité d'attachement préférentiel favorisant les nœuds les moins connectés (les pauvres). Le réseau obtenu est homogène avec une distribution de degré suivant une loi exponentielle, ce qui prouve le rôle crucial de l'effet "rich get richer" sur la topologie du réseau. Alors un traitement préférentiel favorisant les nœuds les moins connectés est équivaut à utiliser une probabilité d'attachement aléatoire. Pour construire un réseau hétérogène, c'est-à-dire une société où peu de gens sont très riches et beaucoup d'autres très pauvres, il faut absolument favoriser les riches.



Dans le troisième chapitre, nous avons étudié en détail la structure  des réseaux sans échelle aléatoires non corrélés. Notre calcul analytique donne de très bons résultats et permet d'obtenir les expressions explicites pour la distribution de la distance (c'est-à-dire le nombre de nœuds se trouvant à une distance donnée d'un nœud racine) et pour le plus court chemin. Il est à noter qu'il n’existe que notre expression explicite pour la distribution de la distance dans le cas des réseaux sans échelle. Nos expressions pour différentes valeurs de l'exposant de la distribution sont en bon accord avec les simulations. Nous avons pu déduire les expressions explicites du plus court chemin dans les réseaux sans échelle pour différentes valeurs de l'exposant de la distribution. Notre principale contribution à cette question est la déduction du plus court chemin lorsque le réseau est très hétérogène, c'est-à-dire  lorsque l'exposant de la distribution est compris entre $2$ et $3$. En effet, il n’existe pas d’équation explicite dans la littérature pour cet intervalle de l'exposant, et seule la forme de mise à l'échelle est donnée. Pour les autres gammes de l'exposant $\gamma$ ($\geq 3$) les expressions obtenues reproduisent les résultats connues. 
Nous avons également montré que lorsque $2 <\gamma <3$, le degré moyen de voisins du nœud ($\kappa$) doit être dépendant de la couche (Fig.~\ref{verefication-kappa}(a)). C’est une propriété cruciale à prendre en compte lorsqu’il s’agit de l’hétérogénéité du réseau.


Dans le quatrième chapitre, en utilisant la transformation de groupe de renormalisation sur le modèle NW, nous avons traité ce modèle de façon plus profond en séparant les nœuds ``aléatoires'' et les nœuds ``réguliers''. Les résultats trouvés sont intéressants et même parfois surprenants. Au début nous avons calculé la fraction des nœuds ``réguliers''  ($S_r$) et ``aléatoires'' ($S_{al}$), puis en s'appuyant sur les fluctuations de $\frac {S_r}{n}$ qu'on a considéré comme paramètre d'ordre, on a conclu qu'une transition de phase de grand monde vers petit monde n'existe pas, après nous avons déduit une expression analytique du plus court chemin, $\textless {\ell} \textgreater$, plus perfectionnée par rapport à l'existant. En fin à partir de cette nouvelle expression de $\textless {\ell} \textgreater$ nous avons conclu que l'invalidité de la fonction universelle de Newman et al. \cite{Newman-Watts1999-2} se révèle  plus ou moins tandis qu'une nouvelle fonction se comporte inversement. Autrement dit la fonction universelle de Newman et al. \cite{Newman-Watts1999-2} n'est valable que lorsque le réseau est grand monde, ce qui se traduit dans nos équations par $y=2k^2\phi \ll 1$. Lorsque le réseau est petit monde, nous avons introduit une nouvelle fonction universelle qui s'écrit en fonction de $y$.


Dans le cinquième chapitre, nous avons essayé d'introduire une méthode pour prédire le type de la transition de phase dans le phénomène de percolation au voisinage de point critique. Le but principal est de spécifier les paramètres adéquats pour réaliser cette prédiction. En effet, nous avons introduit un nouveau paramètre ($\beta'$) qui est l'exposant  effectif de la distribution effectif des amas, ce paramètre représente les corrélations entre les amas, ou plutôt, le type de sélection entre eux. En basant sur ce paramètre et  sur l'exposant de la distribution des amas usuel ($\beta$) nous avons élaboré l'expression générale de la composante géante avec la quelle on peut déduire le type de la transition de phase. Cependant l'expression trouvé n'a pas une forme explicite pour le moment, alors nous avons obligé de faire appel aux approximations et aux arguments moins solides mais plus ou moins convaincant. \\
Nous avons trouvé qu'une transition de phase du premier ordre est impossible si l'exposant de la distribution des amas $\beta>2$, quel que soit le type de comportement collectif entre les amas  dans le réseau, c'est-à-dire indépendant de $\beta'$. En revanche elle est du premier ordre si $\beta\leq2$. De façon générale, il s'avère qu'il suffit de connaître $\beta$ au voisinage du point critique pour en déduire le type de transition de phase.\\  