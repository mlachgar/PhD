% contenu du fichier : conclu.tex
%
\chapter*{Conclusion}
% pour faire apparaitre l'introduction dans le sommaire
\addcontentsline{toc}{chapter}{Conclusion}
 
% Pour que l'entete soit correcte car chapter* ne redefinit pas l'entete.
\markboth{CONCLUSION}{}


Dans cette thèse, quatre contributions ont été effectué au domaine des réseaux complexes, soit par résoudre des problèmes et des difficultés concernant les calculs théoriques, ou par proposer des nouvelles idées. Dans la première contribution, deuxième chapitre, nous avons introduit un simple modèle de réseau complexe avec un critère d'attachement préférentiel sans l'effet "rich get richer", le réseau obtenu est homogène, ce qui démontre le rôle crucial de l'effet "rich get richer" dans la topologie du réseau, ainsi on a déduit qu'un traitement préférentiel aux nœuds les moins connectés équivaut à utiliser une probabilité d'attachement aléatoire. En calculant le degré moyen instantané d'un nœud sélectionné et ses fluctuations, nous montrons comment le degré moyen de hubs et ses fluctuations divergent avec le temps dans le modèle BA, et restent finis dans notre modèle.\\

Dans la deuxième contribution, troisième chapitre, nous avons étudié en détail certains aspects fondamentaux des réseaux libre-échelle aléatoires non corrélés. Avec des étapes et des hypothèses simples, nous avons obtenu les expressions explicites du nombre des nœuds à une distance donnée d'un nœud arbitraire, \nl\nolinebreak. Nous avons obtenu également la description précise de la forme de la distribution \nolinebreak, en plus de détaille, nous avons montré que \nl augmente en tant que loi de puissance pour les premières couches et après avoir atteint un maximum il diminue exponentiellement dans les dernières couches. Profitant de la forme de la distribution \nl, nous avons pu déduire l'expression explicite de PCC. Les expressions obtenues reproduisent les formes de mise à l'échelle connues pour différentes plages de $\gamma$. Autrement dit, le monde ultra-petit pour $2<\gamma<3$, et le petit monde pour $\gamma\ge 3$. Nos résultats théoriques concordent très bien avec les simulations, sauf dans le cas de $\gamma=3$, où nous avons observé la même forme, dans les queues de \nl, mais avec une petite déviation dans la position du maximum. Cette différence n'affecte pas la forme de mise à l'échelle de PCC pour cette valeur de $\gamma$, ainsi que nos expressions sont les plus précises dans la littérature.\\

Dans la troisième contribution, quatrième chapitre, en utilisant la transformation de groupe de renormalisation GR sur le modèle NW, on a trouvé une expression analytique de PCC  mieux que l'ancien déjà existé dans la littérature. \`{A} partir de cette nouvelle expression, on a déduit  à la fois que le paramètre $y=2k^2\phi$ est le variable dont la propriété petit-monde émerge, avec lequel on détermine l'erreur de la fonction universelle $\ell=\acute{n}f(x)$, et en fonction de lui une nouvelle fonction universelle apparaît. Alors c'est une émergence spectaculaire qui se passe en fonction du paramètre $y$.\\

Dans la quatrième contribution, dernier chapitre, nous avons introduit une méthode pour anticiper  le type de transition de phase dans le phénomène de percolation au voisinage de point critique, nous avons trouvé qu'une transition de phase de première ordre est impossible si l'exposant de distribution d'amas $\gamma>2$, indépendamment de la valeur de l'exposant $\gamma'$, en revanche elle est du première ordre si $\gamma\leq2$. De façons générale, il s'avère que si on connaît $\gamma$ au voisinage de point critique, c'est suffisant pour connaître le type de transition de phase.\\  

 