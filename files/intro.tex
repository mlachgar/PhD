
	\newcommand{\nl}{$n_{\hat{\ell}}$ }
\chapter*{Introduction}
% pour faire apparaitre l'introduction dans le sommaire et que les minitocs soient au bon
% endroit
\addstarredchapter{Introduction}  
% Pour que l'entete soit correcte car chapter* ne redefinit pas l'entete.
\markboth{INTRODUCTION}{}
Un réseau complexe est un grand nombre des nœuds liés entre eux selon des topologies de connexion spécifiques. Les réseaux complexes sont omniprésents dans le monde naturel et artificiel grâce aux développements technologiques des $60$ dernières années. En effet,  la plupart des réseaux réels peuvent être représentés par des réseaux complexes, on peut distinguer plusieurs types de réseaux dans les différents domaines, tels que le domaine social, technologique, biologique, physique, etc. Par exemple, un système qui se compose de différents types de molécules qui s'affectent les unes aux autres par des réactions enzymatiques \cite{Je-al2000}, Internet  relie un grand nombre de serveurs et d'ordinateurs dans le monde entier échangeant constamment des quantités gigantesques de paquets d'informations \cite{F-al1999}, le World Wide Web est un réseau virtuel de sites Web liés avec des hyperliens \cite{BA1999}, et les réseaux alimentaires relient également, via des relations trophiques, un grand nombre d'espèces interdépendantes \cite{Co-al1990,Pim-al2002}. Bien que l'existence de ces réseaux dans divers domaines était connue depuis longtemps, les physiciens n'ont commencé à s'y intéresser que depuis la découverte de certaines lois universelles communes à différents systèmes réels. Parmi ces lois les plus importantes on cite: 
\textsf{la distribution de degrés} sans échelle \footnote{ sans-échelle signifie qu'il n'y a pas une valeur de degré caractéristique dans le réseau, mais plutôt une distribution des degrés qui suit une loi de puissance.},
%qui rapportée pour la première fois par des mesures d'Internet par Faloutsos et al. []
la propriété \textsf{petit-monde}  
et la valeur élevée du \textsf{coefficient de regroupement} (Clustering). Afin de bien comprendre les comportements de ces réseaux et établir les lois qui les gouvernent, les physiciens, les mathématiciens et les informaticiens se consacrent à développer des modèles théoriques et des techniques permettant de découvrir  et d'analyser les propriétés de ces réseaux qui sont presque partout. Il semble que le début du troisième millénaire va connaître une nouvelle révolution emportée par les principes des réseaux complexes.\\

Il est évident que la physique statistique est l'outil adéquat pour étudier les systèmes ayant un grand nombre d'éléments en interaction. En effet, elle a développé au cours du temps un ensemble de théories et d'outils mathématiques permettant, à partir des comportements microscopiques, de comprendre l'émergence des caractéristiques macroscopiques. En outre la physique statistique est le cadre théorique le plus perfectionné pour étudier les problèmes de transitions de phases et les points critiques, ce qui est incontournable dans l'étude des systèmes complexes. \\

Ce travail de recherche entreprend quelques contributions à la compréhension de quelques aspects fondamentaux des réseaux complexes, en abordant les problèmes concernant la dynamique de la croissance dans le modèle de Barabási-Albert (BA) \cite{BA1999}, la structure des réseaux  sans échelle, l'émergence de la propriété petit-monde dans le modèle de Newman-Watts \cite{Newman-Watts1999-2} (NW), et la caractérisation des transitions de phases de  percolation lors de l'apparition de la composante géante dans les réseaux complexes aléatoires.\\
De façon plus détaillée, cette thèse est constituée de cinq chapitres. Le premier commence par définir et mettre en évidence l'importance de notre cadre de recherche (réseaux complexes), et se termine par un état de l'art de  certaines techniques et concepts fondamentaux pertinents de la physique statistique et de la théorie des graphes. Le deuxième chapitre commence par traiter la notion du réseau en croissance et de l'attachement préférentiel, et propose un  modèle de croissance en utilisant le complément de la probabilité utilisée dans le modèle BA, et finit par développer le calcul de la valeur moyenne du degré sélectionné à chaque instant et sa fluctuation, et qui sont comparées à ceux du modèle BA. Le but étant de faire une comparaison au niveau microscopique entre les deux et de vérifier si la distribution de degrés en loi de puissance persiste en l'absence du mécanisme "rich get richer". Le troisième chapitre a pour but de donner l'expression explicite de la structure des couches dans les réseaux sans échelle non corrélés, puis celle du plus court chemin, en suggérant que le plus court chemin correspond à la distance à laquelle la couche est maximale. Ce chapitre établie également une comparaison avec les anciens travaux théoriques et avec les données réels. Le quatrième chapitre contient une étude détaillée sur le modèle introduit par Newman-Watts en utilisant la transformation de groupe de renormalisation en espace réel, les résultats obtenus sont très intéressants et nous donnent une  perspective plus clair sur l'émergence de la propriété petit-monde dans ce modèle, on trouve à partir d'une prédiction purement théorique qu'en fonction d'un certain paramètre il y a émergence de la phase petit-monde. Nous étudions également la validité  de la fonction universelle usuelle de NW,  et nous introduisons une nouvelle fonction universelle liée à ce paramètre. Le dernier chapitre est consacré à la transition de phase  de percolation,  qui reste un des problèmes ouverts de la physique statistique. En admettant qu'au point critique les grandeurs pertinentes du système suivent toujours une loi de puissance, nous avons établi une formule théorique qui permet de déterminer les conditions requises pour que la transition de phase sera de premier ou de deuxième ordre, indépendamment des mécanismes microscopiques de l'évolution du système. Autrement dit nous avons étudié le phénomène de percolation au voisinage du point critique en suggérant les conditions nécessaires pour obtenir une transition de premier ordre ou de deuxième ordre.\\


Il convient de signaler que dans les quatre derniers chapitres nous avons développé plusieurs programmes en langage de programmation Fortran 90, basés principalement sur la méthode de Monte Carlo et sur les méthodes d'analyse numérique. Ces programmes servent à vérifier nos équations théoriques ou à confirmer les hypothèses énoncées. En outre, afin de pouvoir travailler sur des réseaux de très grande taille (plus de $10^7$), et exécuter un maximum d'itérations, les programmes ont été amplement optimisés et perfectionnés afin qu'ils puissent fonctionner sur une machine ordinaire. 


 
