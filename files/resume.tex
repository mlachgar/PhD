%
% contenu du fichier : intro.tex
%
\chapter*{Résumé}
% pour faire apparaitre l'introduction dans le sommaire et que les minitocs soient au bon
% endroit
%\addstarredchapter{Introduction générale}  
% Pour que l'entete soit correcte car chapter* ne redefinit pas l'entete.
%\markboth{INTRODUCTION}{}
\addstarredchapter{Résumé} 
 La science du réseau complexe est un domaine de recherche fondamental pour modéliser les réseaux artificiels et naturels dans notre monde réel. La découverte
 des propriétés universelles communs, presque entre tous les réseaux réels, tel que la propriété petit-monde et la distribution libre-échelle, a révolutionné la façon
 d’étudier, de modéliser et de traiter ces réseaux.
 Un réseau complexe est un réseau d’interactions entre entités d’où l’émergence
 des propriétés à grande échelle.\\ L'objectif de cette thèse est de donner quelques contributions au domaine, en première lieu, nous présentons l’état de l’art nécessaire aux lecteurs, puis nous analysons le modèle de Barabási et Albert en s'appuyant sur un nouveau modèle ayant le complément de la probabilité du premier modèle. Aussi nous étudions les réseaux libre-échelle non corrélés, en basant sur  des étapes et des hypothèses simples, nous obtenons les expressions explicites du nombre des nœuds à une distance donnée d'un nœud arbitraire, profitant de la forme de l'expression obtenu, nous déduirons l'expression explicite de plus court chemin. Après nous traitons le modèle petit-monde de Newman et Watts en basant sur la transformation de groupe de renormalisation, nous commençons par établir une formule plus précise de la fonction universelle déjà trouvé par Newman, Moore et Watts, puis nous montrons que celle-ci n'est pas toujours vrai, cependant il y a une saturation de raccourcis et une nouvelle fonction universelle qui sont apparaît, ainsi que le variable de cette nouvelle fonction est différents de l'ancien, nous en déduisons que l'émergence de la propriété petit-monde dans ce modèle est spectaculaire. En fin nous proposons une méthode afin de prédire le type des transitions de phases au voisinage de point critique, cette méthode s'appuyant sur deux paramètres, le premier est l'exposant de la distribution d'amas au voisinage de point critique, le deuxième est l'exposant de la distribution d'amas avec lesquels les amas se choisissant, nous obtenons que, au voisinage de point critique le type de transition de phase dépend seulement au premier paramètre. 