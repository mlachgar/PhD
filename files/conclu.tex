% contenu du fichier : conclu.tex
%
\chapter*{Conclusion}
% pour faire apparaitre l'introduction dans le sommaire
\addcontentsline{toc}{chapter}{Conclusion}
 
% Pour que l'entete soit correcte car chapter* ne redefinit pas l'entete.
\markboth{CONCLUSION}{}


Dans cette thèse, quatre contributions ont été réalisées pour le domaine des réseaux complexes, permettant de résoudre des problèmes de calculs théoriques, et de proposer des nouvelles méthodes. Dans la première contribution, nous avons introduit un simple modèle de réseau complexe avec un critère d'attachement préférentiel sans l'effet "rich get richer", le réseau obtenu est homogène, ce qui démontre le rôle crucial de l'effet "rich get richer" dans la topologie du réseau. Nous en avons déduit qu'un traitement préférentiel aux nœuds les moins connectés équivaut à utiliser une probabilité d'attachement aléatoire. En calculant le degré moyen instantané d'un nœud sélectionné et ses fluctuations, nous avons montré comment le degré moyen de hubs et ses fluctuations divergent avec le temps dans le modèle BA, et restent finis dans notre modèle.\\

Dans la deuxième contribution, nous avons étudié en détail certains aspects fondamentaux des réseaux libre-échelle aléatoires non corrélés. Avec des étapes et des hypothèses simples, nous avons obtenu les expressions explicites du nombre des nœuds à une distance donnée d'un nœud arbitraire, \nl\nolinebreak. Nous avons obtenu également la description précise de la forme de la distribution \nolinebreak. Profitant de la forme de la distribution \nl, nous avons pu déduire l'expression explicite du PCC. Les expressions obtenues reproduisent les formes de mise à l'échelle connues pour différentes plages de $\gamma$. Autrement dit, le monde ultra-petit pour $2<\gamma<3$, et le petit-monde pour $\gamma\ge 3$. Nos résultats théoriques concordent très bien avec les simulations, sauf dans le cas de $\gamma=3$, où nous avons observé la même forme, dans les queues de \nl, mais avec une petite déviation dans la position du maximum. Cette différence n'affecte pas la forme de mise à l'échelle du PCC pour cette valeur de $\gamma$, ainsi nos expressions restent les plus précises dans la littérature.\\

Dans la troisième contribution, en utilisant la transformation de groupe de renormalisation GR sur le modèle NW, nous avons élaboré une expression analytique du PCC plus perfectionnée par rapport à l'existant. En effet, à partir de cette nouvelle expression, nous avons montré  que suivant le paramètre $y=2k^2\phi$ la propriété petit-monde émerge, et selon sa valeur l'erreur de la fonction universelle $\ell=\acute{n}f(x)$ se révèle plus ou moins tandis que la nouvelle fonction se comporte inversement. Alors c'est une émergence spectaculaire qui se passe en fonction du paramètre $y$.\\

Dans la quatrième contribution, nous avons introduit une méthode pour anticiper  le type de transition de phase dans le phénomène de percolation au voisinage de point critique, nous avons trouvé qu'une transition de phase du premier ordre est impossible si l'exposant de distribution d'amas $\gamma>2$, quelle que soit les corrélations dans le système donné, en revanche elle est du premier ordre si $\gamma\leq2$. De façons générale, il s'avère qu'il suffit de connaître $\gamma$ au voisinage de point critique pour en déduire le type de transition de phase.\\  
