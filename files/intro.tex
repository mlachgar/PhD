%
% contenu du fichier : intro.tex
%
\chapter*{Introduction}
% pour faire apparaitre l'introduction dans le sommaire et que les minitocs soient au bon
% endroit
\addstarredchapter{Introduction}  
% Pour que l'entete soit correcte car chapter* ne redefinit pas l'entete.
\markboth{INTRODUCTION}{}
Un réseau complexe est un grand nombre des nœuds liés entre eux selon des topologies de connexion spécifiques. Les réseaux complexes sont omniprésents dans le monde naturel et artificiel, cette compréhension est liée aux développements technologiques des $60$ dernières années, en effet,  la plupart des réseaux réels peuvent être représentés par des réseaux complexes, on peut distinguer plusieurs types de réseaux dans les différents domaines, tels que le domaine social, technologique, biologique, physique, etc. Par exemple, un système qui se compose de différents types de molécules qui s'affectent les unes aux autres par des réactions enzymatiques \cite{Je-al2000}, Internet  relie un grand nombre de serveurs et d'ordinateurs dans le monde entier échangent constamment des quantités énormes de paquets d'informations \cite{F-al1999}, le World Wide Web est un réseau virtuel de sites Web liés avec des hyperliens \cite{BA1999}, et les réseaux alimentaires relient également, via des relations trophiques, un grand nombre d'espèces interdépendantes \cite{Co-al1990,Pim-al2002}. Bien que l'existence de ces réseaux dans divers domaines était connue depuis longtemps, les physiciens n'ont commencé à s'y intéresser que depuis la découverte de certains lois universels communs à différents systèmes réels. Parmi ces lois universels les plus importantes on cite: 
\textsf{la distribution de degrés} libre-échelle \footnote{ Libre-échelle signifie qu'il n'y a pas une valeur de degré caractéristique dans le réseau, mais plutôt une distribution des degrés qui suit une loi de puissance.},
%qui rapportée pour la première fois par des mesures d'Internet par Faloutsos et al. []
la propriété \textsf{petit-monde}  
et la valeur élevée de \textsf{coefficient de regroupement} (Clustering). Afin de bien comprendre les comportements de ces réseaux et établir les lois qui les gouvernent, les physiciens, les mathématiciens et les informaticiens se consacrent à développer des modèles théoriques et des techniques permettant de découvrir  et d'analyser les propriétés de ces réseaux qui sont presque partout. Il semble que le début du troisième millénaire va connaître une nouvelle révolution emportée par les principes des réseaux complexes.\\

Il est évident que la physique statistique est l'outil adéquat pour étudier les systèmes ayant un grand nombre d'éléments en interaction. En effet, elle a développé au cours du temps un ensemble de théories et d'outils mathématiques permettant, à partir des comportements microscopiques, de comprendre l'émergence des caractéristiques macroscopiques et d’étudier systématiquement la topologie de ces grands réseaux complexes. En outre la physique statistique est le cadre théorique le plus perfectionné pour étudier les problèmes de transitions de phases et les points critiques, ce qui est incontournable dans l'étude des systèmes complexes. \\


Ce travail de recherche entreprend quelques contributions à la modélisation des réseaux complexes, en abordant quelques problèmes concernant la dynamique de la croissance dans le modèle de Barabási-Albert (BA) \cite{BA1999}, la structure des réseaux  libre-échelle, l'émergence de la propriété petit-monde dans le modèle de Newman-Watts \cite{Newman-Watts1999-2}, et la caractérisation des transitions de phases dans le phénomène de  percolation.

De façon plus détaillée, cette thèse est divisée en cinq chapitres. Le premier commence par définir et mettre en évidence l'importance de notre cadre de recherche (réseaux complexes), et se termine par un état de l'art de  certaines techniques et concepts fondamentaux pertinents de la physique statistique et de la théorie des graphes. Le deuxième chapitre consiste au début à donner la notion du réseau en croissance et de l'attachement préférentiel, puis à proposer un simple modèle en utilisant le complément de la probabilité utilisée dans le modèle BA, enfin de développer le calcul de la valeur moyenne de degré sélectionné à chaque instant et leur fluctuation dans notre modèle et celui de BA, afin de faire une comparaison au niveau microscopique entre les deux et de vérifier si la distribution de degrés en loi de puissance reste en l'absence du mécanisme "rich get richer". Le troisième chapitre a pour but de donner l'expression explicite de la structure des couches dans les réseaux libre-échelle non corrélés, puis celle de plus court chemin, en suggérant que le plus court chemin correspond à la distance à laquelle la couche est maximale. Ainsi qu'une comparaison avec les anciens travaux théoriques et avec les données réels. Le quatrième chapitre contient une étude détaillée sur le modèle introduit par Newman-Watts en utilisant la transformation de groupe de renormalisation en espace réel, les résultats obtenus sont très intéressants et nous donnent une  perspective plus clair sur l'émergence vers la propriété petit-monde dans ce modèle, on trouve à partir d'une prédiction purement théorique\footnote{ 
C'est-à-dire que cette idée est une conclusion qui revient directement de nos équations théoriques et pas d'une manière intellectuelle ou empirique.} qu'en fonction d'un certain paramètre il y a une émergence spectaculaire vers la phase petit-monde, aussi la disparition de la fonction universelle usuelle   et l'apparaît d'une autre nouvelle fonction universelle est liée à ce paramètre, on déduit alors que cette émergence n'est pas ordinaire mais de nature spectaculaire. Le dernier chapitre est consacré au problème de transition de phase dans le phénomène de percolation,  qui est un problème pas encore bien compris et le plus urgent à résolus dans la physique des systèmes désordonnés. Sous la proposition que les lois au point critique sont toujours lois de puissance on a établi une formule théorique avec laquelle on détermine les conditions pour que la transition de phase sera de première ou de deuxième ordre, indépendamment de mécanismes microscopiques de l'évolution de système, autrement dit on a étudié le phénomène de percolation au voisinage de point critique en démontrant théoriquement les conditions nécessaires pour obtenir une transition de première ordre ou de deuxième ordre.\\

Il convient de signaler que dans tous les quatre derniers chapitres on a développé un grand nombre de programmes par Fortran, basant dans la plus part de temps sur la méthode de Monte Carlo et parfois sur les méthodes d'analyse numérique, ces programmes sont souvent pour vérifier nos équations théoriques ou parfois pour confirmer les idées qu'on a proposé. En outre, à cause de la nécessité d'atteindre une grande taille du réseau, parfois plus de $10^7$, et un grand nombre de réalisation, on a fait beaucoup des efforts afin d'optimiser nos programmes.




 