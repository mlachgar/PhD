%
% contenu du fichier : intro.tex
%
\chapter*{Résumé}
% pour faire apparaitre l'introduction dans le sommaire et que les minitocs soient au bon
% endroit
%\addstarredchapter{Introduction générale}  
% Pour que l'entete soit correcte car chapter* ne redefinit pas l'entete.
%\markboth{INTRODUCTION}{}
\addstarredchapter{Résumé} 
 La science des réseaux complexes est un domaine de recherche fondamental qui permet de modéliser et étudier les réseaux artificiels et naturels dans notre monde réel. La découverte
 des propriétés universelles communes, quasiment à tous les réseaux réels, tel que la propriété petit-monde et la distribution libre-échelle, a révolutionné la façon
 d’étudier, modéliser et traiter ces réseaux.
 Un réseau dit complexe est constitué d'un grand nombre d'entités en interactions d’où l’émergence
 des propriétés à grande échelle.\\ L'objectif de cette thèse est d'apporter quelques contributions à ce domaine. En premier lieu, nous présenterons l’état de l’art nécessaire aux lecteurs, puis nous analyserons le modèle de Barabási et Albert en s'appuyant sur un nouveau modèle ayant le complément de la probabilité du premier. Puis nous étudierons les réseaux libre-échelle non corrélés, en se basant sur des étapes et des hypothèses simples, nous obtiendrons les expressions explicites du nombre de nœuds à une distance donnée d'un nœud arbitraire, profitant de la forme du résultat obtenu, nous déduirons l'expression explicite du plus court chemin. Ensuite nous traitons le modèle petit-monde de Newman et Watts en se basant sur la transformation de groupe de renormalisation, nous commencerons par établir une formule plus précise de la fonction universelle de Newman, Moore et Watts, puis nous montrerons que celle-ci n'est pas toujours vraie. En effet nous constaterons une saturation de raccourcis et l'apparition d'une nouvelle fonction universelle dont les variables sont différentes.Nous en déduisons que l'émergence de la propriété petit-monde dans ce modèle est spectaculaire. Enfin nous proposerons une méthode permettant de prédire le type des transitions de phases au voisinage du point critique, cette méthode s'appuyant sur deux paramètres, le premier est l'exposant de la distribution d'amas au voisinage du point critique, le second est l'exposant de la distribution d'amas avec lesquels les amas se choisissent, nous conclurions qu'au voisinage du point critique le type de transition de phase dépend uniquement du premier paramètre. 